\section{Open Questions and Issues}

\begin{itemize}
\item Equations: The coupling parameters $\alpha$ and $\kappa$ must be available to both the connection between a node's \textsf{PE unit} and the parent's \textsf{UPDATE unit}, as well as the connection between a node's \textsf{PREDICTION unit} and the parent's \textsf{UPDATE unit}. How can it be ensured that the coupling strength is the same for both connections?

\item Equations: In the special case of a node that has both a \textsf{VAPE} and a \textsf{VOPE} parent, the PE that is sent to the \textsf{VOPE} parent also depends on the posterior(?) precision of the \textsf{VAPE} parent. This could be another typo, but if not, I cannot see how this could be imlemented.
\begin{equation}
\delta_i^{(k)} 	= \frac{\frac{1}{\pi_i^{(k)}}
				+ \alpha_{i,i+1}^2 \frac{1}{\pi_{i+1}^{(k)}}
				+ (\mu_i^{(k)} - \hat{\mu}_i^{(k)})^2}
				{\frac{1}{\pi_i^{(k-1)}} + \nu_i^{(k)}} - 1
\end{equation}

\item Implementation: Would we really distinguish between neurons signalling the posteriors (beliefs; $\mu_i$ and $\pi_i$) and those signalling the priors (predictions; $\hat{\mu}_i$ and $\hat{\pi}_i$)? If we don't, then implementation gets more complicated, as nodes have to have access to their previous states or beliefs.

\item Implementation: To whom does the knowledge about parameters, in particular coupling strengths, belong? Ideally, this would belong to the connections themselves and be accessible to both child and parent node. However, in terms of synaptic signalling, how can a node have direct access to a coupling strength other than via the signalled value (which is the product of the coupling strength and the signal)?

\item Not covered here: Computations of input nodes and binary input nodes.
\item Not covered here: Equations for multiple children and parent nodes, except for the special cases described above.
\end{itemize}