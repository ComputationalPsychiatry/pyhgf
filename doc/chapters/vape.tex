\section{Computations for VAPE coupling}
The exact computations of the \textsf{UPDATE step} depend on the nature of the coupling with the child node(s), while both the \textsf{PE step} and the \textsf{PREDICTION step} depend on the coupling with the parent node(s).

\subsection{Update Step}
If Node~$i$ is the value parent of Node $i-1$, then the following update equations apply to Node~$i$:
\vspace{0.5cm}

\noindent
\fbox{
\begin{minipage}{\textwidth}

\begin{align}
\pi_i^{(k)} &= \hat{\pi}_i^{(k)} 
			+ \alpha_{i-1,i}^2 \hat{\pi}_{i-1}^{(k)}\\
\mu_i^{(k)} &= \hat{\mu}_i^{(k)} 
			+ \frac{\alpha_{i-1,i}^2 \hat{\pi}_{i-1}^{(k)}} {\alpha_{i-1,i}^2 \hat{\pi}_{i-1}^{(k)} + \hat{\pi}_{i}^{(k)}} \delta_{i-1}^{(k)}
\end{align}			
\vspace{1pt}

\end{minipage}
}%end:fbox
\vspace{0.5cm}

\noindent
We note here that we can let the update of the precision happen first, and therefore use it for the update of the mean:
\vspace{0.5cm}

\noindent
\fbox{
\begin{minipage}{\textwidth}

\begin{align}
\pi_i^{(k)} &= \hat{\pi}_i^{(k)} 
			+ \alpha_{i-1,i}^2 \hat{\pi}_{i-1}^{(k)}\\
\mu_i^{(k)} &= \hat{\mu}_i^{(k)} 
			+ \frac{\alpha_{i-1,i}^2 \hat{\pi}_{i-1}^{(k)}} {\pi_i^{(k)}} \delta_{i-1}^{(k)}
\end{align}			
\vspace{1pt}

\end{minipage}
}%end:fbox
\vspace{0.5cm}

\noindent
In sum, at the time of the update, Node~$i$ needs to have access to the following quantities:

\begin{description}
\item[Its own predictions:]  	$\hat{\mu}_i^{(k)}$, $\hat{\pi}_i^{(k)}$
\item[Coupling strength:] 		$\alpha_{i-1,i}$
\item[From level below:]		$\delta_{i-1}^{(k)}$, $\hat{\pi}_{i-1}^{(k)}$
\end{description}

All of these are available at the time of the update. Node~$i$ therefore only needs to receive the PE and the predicted precision from the level below to perform its update.

\subsection{Prediction Error Step}
We will assume in the following, that Node~$i$ is the value child of Node $i+1$. Then the following quantities have to be sent up to Node $i+1$ (cf. necessary information from level below in a value parent):

\begin{description}
\item[Predicted precision:] 	$\hat{\pi}_{i}^{(k)}$
\item[Prediction error:]		$\delta_{i}^{(k)}$
\end{description}

Node~$i$ has already performed the \textsf{PREDICTION step} on the previous trial, so it has already computed the predicted precision of the current trial,~$\hat{\pi}_{i}^{(k)}$. Hence, in the \textsf{PE step}, it needs to perform only the following calculation:
\vspace{0.5cm}

\noindent
\fbox{
\begin{minipage}{\textwidth}

\begin{equation}
\delta_i^{(k)} 	= \mu_i^{(k)} - \hat{\mu}_i^{(k)}
\end{equation}			
\vspace{1pt}

\end{minipage}
}%end:fbox
\vspace{0.5cm}

\noindent


\subsection{Prediction Step}
Still assuming that Node~$i$ is the value child of Node $i+1$, the \textsf{PREDICTION step} consists of the following computations:
\vspace{0.5cm}

\noindent
\fbox{
\begin{minipage}{\textwidth}

\begin{align}
\hat{\mu}_i^{(k+1)} 	&= \mu_i^{(k)} + \alpha_{i,i+1} \mu_{i+1}^{(k)}\\
\hat{\pi}_i^{(k+1)} 	&= \frac{1}{\frac{1}{\pi_i^{(k)}} + \nu_i^{(k+1)} }
\end{align}			
\vspace{1pt}

\end{minipage}
}%end:fbox
\vspace{0.5cm}

\noindent
with
\begin{equation*}
\nu_i^{(k+1)} = \exp(\omega_i)).
\end{equation*}

Note that if Node~$i$ additionally has a \textsf{VOPE} parent node, the estimated volatility $\nu_i^{(k+1)}$ that enters the precision update would also depend on the posterior mean of that volatility parent (cf. \textsf{PREDICTION step} for \textsf{VOPE} coupling).\\

In general, the prediction of the mean will depend only on whether Node~$i$ has a value parent or not, whereas the prediction of the precision only depends on whether Node~$i$ has a volatility parent or not. \\

However, without a volatility parent node, the estimated volatility only depends on the node's own learning rate $\omega_i$, i.e., it is a constant. We can therefore simply write:
\vspace{0.5cm}

\noindent
\fbox{
\begin{minipage}{\textwidth}

\begin{align}
\hat{\mu}_i^{(k+1)} 	&= \mu_i^{(k)} + \alpha_{i,i+1} \mu_{i+1}^{(k)}\\
\hat{\pi}_i^{(k+1)} 	&= \frac{1}{\frac{1}{\pi_i^{(k)}} + \exp(\omega_i) }
\end{align}			
\vspace{1pt}

\end{minipage}
}%end:fbox
\vspace{0.5cm}

\noindent

Thus, the \textsf{PREDICTION step} only depends on knowing the node's own posteriors and receiving the value parent's posterior in time before the new input arrives. \\

In general, the prediction of the mean will depend only on whether Node~$i$ has a value parent or not, whereas the prediction of the precision only depends on whether Node~$i$ has a volatility parent or not. \\

The \textsf{PREDICTION step} could also be placed in the beginning of a trial. However, we usually think about the beginning of a trial as starting with receiving a new input, and of a prediction as being present before that input is received. It would probably make more sense to think of the prediction as happening (continuously) between trials. For implementational purposes, it is important to keep in mind that the posterior means of parent nodes have to be sent back to their children once they're computed, such that children nodes can prepare for the next input in time.

%\newpage